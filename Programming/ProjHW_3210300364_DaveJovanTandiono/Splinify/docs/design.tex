\documentclass[10pt,a4paper]{article}

\usepackage[utf8]{inputenc}       
\usepackage{amsmath, amssymb}     
\usepackage{geometry}             
\usepackage{float}                
\usepackage{listings}             
\usepackage{booktabs}             
\usepackage{hyperref}             
\usepackage{xcolor}               
\geometry{a4paper, margin=1in}

\lstset{
  basicstyle=\ttfamily\small,
  breaklines=true,
  frame=single,
  captionpos=b,
  keywordstyle=\color{blue},
  stringstyle=\color{red},
  commentstyle=\color{gray},
}

\title{\textbf{Programming Project Design}}
\author{DAVE JOVAN TANDIONO 3210300364 \\ 
\textit{Information and Mathematical Science 2101, Zhejiang University} \\ 
Due time: January 1, 2025}
\date{}

\begin{document}

\maketitle

\section*{Abstract}
This report presents a comprehensive design and implementation of the Splinify system, a computational tool for spline interpolation. The document elaborates on the mathematical principles, software architecture, implementation details, and test results for various types of splines, including cubic splines, B-splines, and piecewise polynomial splines.

\tableofcontents

\section{Introduction}
Spline interpolation is a mathematical technique to approximate functions or data points using piecewise polynomials. The Splinify system is designed to implement and test different spline types, focusing on:
\begin{itemize}
    \item Natural and clamped cubic splines
    \item B-splines with different boundary conditions
    \item Piecewise polynomial splines
\end{itemize}

The system is implemented in C++ and includes Python scripts for post-processing.

\section{Mathematical Formulation}
This section describes the mathematical foundation for the splines implemented in the Splinify system.

\subsection{Cubic Spline}
A cubic spline is a piecewise function $S(x)$, defined as:
\[
S(x) = a_i + b_i(x - x_i) + c_i(x - x_i)^2 + d_i(x - x_i)^3, \quad x \in [x_i, x_{i+1}]
\]
where the coefficients $a_i, b_i, c_i, d_i$ are determined by:
\begin{enumerate}
    \item Continuity at internal points
    \item Smoothness conditions for the first and second derivatives
    \item Boundary conditions:
    \begin{itemize}
        \item \textbf{Natural Boundary:} $S''(x_0) = 0$ and $S''(x_n) = 0$
        \item \textbf{Clamped Boundary:} Specified values for $S'(x_0)$ and $S'(x_n)$
    \end{itemize}
\end{enumerate}

\subsection{B-Spline}
A B-spline is defined using basis functions $B_{i,k}(x)$:
\[
B_{i,1}(x) = 
\begin{cases} 
1 & \text{if } x \in [x_i, x_{i+1}) \\
0 & \text{otherwise}
\end{cases}
\]
For $k > 1$, the recursive relation is:
\[
B_{i,k}(x) = \frac{x - x_i}{x_{i+k-1} - x_i}B_{i,k-1}(x) + \frac{x_{i+k} - x}{x_{i+k} - x_{i+1}}B_{i+1,k-1}(x)
\]

\subsection{Piecewise Polynomial}
A piecewise polynomial is constructed as:
\[
P(x) = \begin{cases}
P_1(x) & x \in [x_0, x_1] \\
P_2(x) & x \in [x_1, x_2] \\
\vdots \\
P_n(x) & x \in [x_{n-1}, x_n]
\end{cases}
\]

\section{Implementation}
The Splinify system is implemented in C++ with Python for post-processing and testing. The key modules are:
\begin{itemize}
    \item \textbf{SplineUtils:} Includes core classes like \texttt{CubicSpline}, \texttt{BSpline}, and \texttt{PiecewisePolynomialSpline}.
    \item \textbf{InputValidator:} Ensures valid input data.
    \item \textbf{Visualization:} Python scripts to generate numerical outputs.
\end{itemize}

\subsection{Code Structure}
The directory structure is as follows:
\begin{verbatim}
Splinify/
  |- src/
     |- SplineUtils/
  |- tests/
     |- CppTests/
  |- scripts/
  |- docs/
\end{verbatim}

\subsection{Code Example}
The following code computes a cubic spline:
\begin{lstlisting}[language=C++]
#include <iostream>
#include "CubicSpline.h"

int main() {
    CubicSpline spline({0, 1, 2}, {0, 1, 0});
    double y = spline.interpolate(1.5);
    std::cout << "Interpolated value at x=1.5: " << y << std::endl;
    return 0;
}
\end{lstlisting}

\section{Testing and Results}
The Splinify system was tested with multiple datasets.

\subsection{Cubic Spline Test}
The cubic spline test produced the following results:
\[
\begin{array}{|c|c|}
\hline
x & y \\
\hline
0 & 0 \\
0.5 & 1.01562 \\
1 & 2 \\
1.5 & 1.48437 \\
2 & 1 \\
\hline
\end{array}
\]

\subsection{B-Spline Test}
The B-spline test confirmed smoothness at control points, with numerical outputs verifying the accuracy.

\subsection{Piecewise Polynomial Test}
The piecewise polynomial spline was evaluated for edge cases, ensuring correct interpolation for non-uniform datasets.

\section{Future Work}
The following enhancements are proposed for the Splinify system:
\begin{itemize}
    \item Extend to higher-dimensional splines for 3D data visualization.
    \item Optimize computational efficiency for handling large datasets.
    \item Integrate machine learning to predict optimal spline parameters.
\end{itemize}

\section{Conclusion}
The Splinify system provides a robust framework for implementing and testing spline interpolation techniques. The integration of numerical analysis and computational efficiency makes it a versatile tool for data approximation tasks.

\bibliographystyle{plain}
\bibliography{references}

\end{document}